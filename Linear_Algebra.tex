\documentclass[UTF8]{ctexart}
    \usepackage{amsmath}
    \usepackage{hyperref}
    \usepackage{indentfirst}
    \usepackage{cite}
    \usepackage{titling}
    \usepackage{multicol}
    \usepackage{geometry}
    \usepackage[super,square]{natbib}
    \geometry{left=2cm,right=2cm,top=2.5cm,bottom=2cm}
    \setlength{\parindent}{2em}
    \title{高等代数复习笔记}
    \date{}
    \begin{document}
    \maketitle
    \pagestyle{plain}
    \hrule
    \tableofcontents
    \hrule
    \section{线性方程组求解}
    \subsection{矩阵消元}不再赘述。
    \subsection{$Ax=b$}
    \subsubsection{解的存在性}
    当系数矩阵的秩与增广矩阵的秩相等时,有解。即
    \[\displaystyle R(A,b)=R(A) \]
    \indent
    而对于$A\in F^{n\times n},Ax=b$有唯一解,等价于$|A|\neq 0$\\
    \indent
    考虑$Ax=0,V=\left\{{x|Ax=0}\right\}$\\
    \indent
    有$dim(V)+Rank(A)=n$(秩-零化度定理)\\
    \indent
    同时,有$A=(A_1...A_n)^T,V(A_1...A_n)\perp V$\\
    \indent
    对于$Ax=b$,解可写成一个特解加上解$Ax=0$得到的通解。
    \section{向量空间}
    \subsection{八个运算律(以下均为缩略表示)}
    \subsubsection{加法交换律}$a+b=b+a$
    \subsubsection{加法结合律}$(a+b)+c=a+(b+c)$
    \subsubsection{存在零向量}$0=(0,...,0),a+0=0+a$
    \subsubsection{存在负向量}$对于a=(a_1,...,a_n),存在负向量-a=(-a_1,...,-a_n),满足a+(-a)=0$
    \subsubsection{数乘对于数的加法的分配律}$(\lambda+\mu)a=\lambda a+\mu a$
    \subsubsection{数乘对于向量加法的分配律}$\lambda(a+b)=\lambda a+\lambda b$
    \subsubsection{数乘结合律}$\lambda(\mu a)=(\lambda \mu)a$
    \subsubsection{1乘向量}$1a=a$
    \subsection{线性相关与线性无关}
    对于n个向量$\alpha_1,...,\alpha_2$,考虑
    \[\displaystyle \lambda_1\alpha_1+...+\lambda_n\alpha_n=0\]
    \indent
    若上式当且仅当$\lambda_1,...,\lambda_n$全为0时才满足,则称这n个向量线性无关,否则线性相关。
    \subsection{线性组合}
    具体含义不再赘述。\\
    \indent
    若B是A的线性组合,则$Rank(A)\geq Rank(B)$
    \subsection{秩}
    对于一个向量组,它的秩就是极大线性无关向量组中的向量的个数。不再赘述。
    \subsection{基}
    基的概念由秩派生而来,不再赘述。
    \subsubsection{坐标变换}
    坐标:有序数组$(x_1,...,x_n)$,通常记作列向量。即$\alpha=(a_1,...,a_n)(x_1,...,x_n)^T$\\
    \indent
    使用NB代表新基,OP代表旧坐标,NP代表新坐标,则有$[NB][NP]=[OP]$,而通过化简$[NB,OP]$可得到NP。
    \subsubsection{过渡矩阵}
    过渡矩阵是基与基之间的可逆线性变换,具体请参照教材,不再赘述。\\
    \indent
    由基A变到基B:$B=AP$,由坐标X(A)变到坐标Y(B):$X=PY$,使用逆运算即可求解处过渡矩阵。
    \subsection{子空间}
    子空间W是数域F上的向量空间V的非空子集,且满足加法与数乘封闭。\\
    \indent
    子空间W包含的线性无关向量的最大个数称为W的维数,记作dimW。其余可参照教材,不再赘述。
    \subsubsection{子空间的交与和}
\end{document}